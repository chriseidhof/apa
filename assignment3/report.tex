\documentclass[a4wide,12pt]{article}
\usepackage{a4wide}
\usepackage{times}
\usepackage{fancyvrb}
\usepackage{url}
\usepackage{enumerate}
\usepackage{palatino}
\usepackage{rotating}
 
\usepackage{prooftree}

\usepackage{amsmath, amsthm, amssymb}

\theoremstyle{definition}
\newtheorem{defi}{Definition}
\newtheorem{example}{Example}
\newtheorem*{conj}{Conjecture}
\newtheorem*{prob}{Problem}
\newtheorem*{question}{Question}
\theoremstyle{plain} 
\newtheorem{theo}{Theorem}
\newtheorem{prop}[theo]{Proposition}
\newtheorem{lemma}[theo]{Lemma}
\newtheorem{cor}[theo]{Corollary}
\newtheorem*{theo*}{Theorem}
\newtheorem*{prop*}{Proposition}
\newtheorem*{lemma*}{Lemma}
\newtheorem*{cor*}{Corollary}
\theoremstyle{remark}
\newtheorem*{remark}{Remark}
\newtheorem*{notation}{Notation}
\def\qed{\begin{flushright} $\Box$ \end{flushright}}

\newenvironment{prf}
                {\vspace{-2mm} \noindent {\bf Proof.}}
                {\par \nopagebreak \qed }
\newenvironment{namedprf}[1]
                {\noindent {\bf Proof (#1).}}
                {\par \nopagebreak \qed }


\def\logequiv{\Leftrightarrow}

\allowdisplaybreaks[0]
 
\def\eq{\;\; = \;\;}
\def\N{\mathbb{N}}
\def\Z{\mathbb{Z}}
 
\def\pset#1{\mathcal{P}(#1)}
\def\A#1{\mathcal{A}[\hspace{-1pt}[#1]\hspace{-1pt}]}
 
\def\const#1{\mathopen{\langle}#1\mathclose{\rangle}} % <a,b,...z>
\def\pair#1{\const{#1}}
 
\def\Stmt {\mathbf{Stmt}}
\def\Lab {\mathbf{Lab}}
\def\Blocks{\mathbf{Blocks}}
\def\Var {\mathbf{Var}}
 
 
\def\skip {\texttt{skip}\ }
\def\whilel{\texttt{while}\ }
\def\dol {\texttt{do}\ }
\def\ifl {\texttt{if}\ }
\def\thenl {\texttt{then}\ }
\def\elsel {\texttt{else}\ }
\def\printl{\texttt{print}\ }
\def\contl {\texttt{continue}\ }
\def\breakl{\texttt{break}\ }
 
\def\haskell{\textsc{Haskell}}
\def\starto{\overset{\star}{\to}}
 
\newcounter{Progenvcount}
\setcounter{Progenvcount}{0}
\newenvironment{progenv}
                {\refstepcounter{Progenvcount} \nopagebreak
                 \bigskip\hrule\nopagebreak\medskip\noindent
                 {\bf Program \arabic{Progenvcount}.} \nopagebreak\vspace{0.3cm} \nopagebreak \\ \nopagebreak
                  \nopagebreak
                 $\begin{array}{ll}}
                {\end{array}$ \bigskip\hrule\bigskip\bigskip }


%\def\programold#1{\fbox{\begin{minipage}{0.5\textwidth}\protect{$\begin{array}{ll} #1 \end{array}$}\end{minipage}}}

\def\restabR#1#2[#3]{
\begin{table}
{\scriptsize
\caption{#1}\label{#3}
\begin{center}\begin{sideways}\input{#2}\end{sideways}\end{center}
}
\end{table}}

\def\restabRtiny#1#2[#3]{
\begin{table}
{\tiny
\caption{#1}\label{#3}
\begin{center}\begin{sideways}\input{#2}\end{sideways}\end{center}
}
\end{table}}


\def\restab#1#2[#3]{
\begin{table}
\caption{#1}\label{#3}
\begin{center}\input{#2}\end{center}
\end{table}}

\def\sqleq{\sqsubseteq}

\def\htau{\hat{\tau}}
\def\HGamma{\hat{\Gamma}}
\def\judge#1#2#3{#1 \dashv #2 : #3}

\def\annot#1{\|#1\|}

\def\program#1[#2]{\begin{progenv}\label{#2}\input{#1}\end{progenv}}

\def\Tiny{\fontsize{3pt}{3pt}\selectfont}
\def\hs#1{\texttt{#1}}
 
\begin{document}
\author{Chris Eidhof, Rui S. Barbosa}
\title{Type and Effect Systems Assignment \\ Automatic Program Analysis}
 
\maketitle

\section{Monovariant Binding Time Analysis}

types:
\[\tau := bool | int | \tau_1 \to \tau_2\]

annotated types:
\begin{eqnarray*}
\htau & := bool^\varphi | int^\varphi | \htau_1 \overset{\varphi}{\to} \htau_2 
\\
\varphi & := S | D
\end{eqnarray*}

Given an annotated type $\htau$, $\annot{\htau}$ stands for its topmost annotation.
That is:
\begin{eqnarray*}
\annot{bool^\varphi} & = \varphi \\
\annot{int^\varphi} & = \varphi \\
\annot{\htau_1 \overset{\varphi}{\to} \htau_2} & = \varphi \\
\end{eqnarray*}

\begin{eqnarray*}
\lbrack con \rbrack\;\; &
\begin{prooftree}
\justifies
\judge{\HGamma}{c}{\tau_c^\varphi}
\end{prooftree}\\
& & \\
\lbrack var\rbrack\;\; &
\begin{prooftree}
\HGamma(x)=\htau
\justifies
\judge{\HGamma}{x}{\htau}
\end{prooftree}\\
& & \\
\lbrack fn \rbrack\;\; &
\begin{prooftree}
\judge{\HGamma[x\mapsto\htau_1]}{e}{\htau_2}
\justifies
\judge{\HGamma}{fn x \Rightarrow e}{\htau_1 \overset{\varphi}{\to} \htau_2}
\end{prooftree} if
\varphi \sqleq \annot{\htau_1} \land \varphi \sqleq \annot{\htau_2} \\
& & \\
\lbrack fun \rbrack\;\; &
\begin{prooftree}
\judge{\HGamma[f \mapsto \htau_1 \overset{\varphi}{\to} \htau_2][x\mapsto\htau_1]}{e}{\htau_2}
\justifies
\judge{\HGamma}{fun x \Rightarrow e}{\htau_1 \overset{\varphi}{\to} \htau_2}
\end{prooftree} if
\varphi \sqleq \annot{\htau_1} \land \varphi \sqleq \annot{\htau_2} \\
& & \\
\lbrack app \rbrack\;\; &
\begin{prooftree}
\judge{\HGamma}{e_1}{\htau_1 \overset{\varphi}{\to} \htau_2}  \judge{\HGamma}{e_2}{\htau_1}
\justifies
\judge{\HGamma}{e_1 \; e_2}{\htau_2}
\end{prooftree}\\
& & \\
\lbrack if \rbrack\;\; &
\begin{prooftree}
\judge{\HGamma}{e_0}{bool^\varphi_0}
\judge{\HGamma}{e_1}{\htau}
\judge{\HGamma}{e_2}{\htau}
\justifies
\judge{\HGamma}{if e_0 then e_1 else e_2}{\htau}
\end{prooftree}if
\varphi_0 \sqleq \annot{\htau}\\
& & \\
\lbrack let \rbrack\;\; &
\begin{prooftree}
\judge{\HGamma}{e_1}{\htau_1}
\judge{\HGamma[x \mapsto \htau_1]}{e_2}{\htau_2}
\justifies
\judge{\HGamma}{let x=e_1 in e_2}{\htau_2}
\end{prooftree}\\
& & \\
\lbrack op \rbrack\;\; &
\begin{prooftree}
\judge{\HGamma}{e_1}{\htau_op_1^\varphi}
\judge{\HGamma[x \mapsto \htau_1]}{e_2}{\htau_op_2^\varphi}
\justifies
\judge{\HGamma}{let x=e_1 in e_2}{\htau_op^\varphi}
\end{prooftree}\\
& & \\
\end{eqnarray*}


ALGORITHM W

\section{Polyvariant}
\section{Subtyping}

\end{document}  

