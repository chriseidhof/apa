\documentclass[a4wide,12pt]{article}
\usepackage{a4wide}
\usepackage{times}
\usepackage{fancyvrb}
\usepackage{url}
\usepackage{enumerate}
\usepackage{palatino}
\usepackage{rotating}
 
\usepackage{prooftree}

%include lhs2tex.fmt 
 
\usepackage{amsmath,amsthm,amssymb}

 
\def\eq{\;\; = \;\;}
\def\N{\mathbb{N}}
\def\Z{\mathbb{Z}}
 
\def\pset#1{\mathcal{P}(#1)}
\def\A#1{\mathcal{A}[\hspace{-1pt}[#1]\hspace{-1pt}])}
 
\def\const#1{\mathopen{\langle}#1\mathclose{\rangle}} % <a,b,...z>
\def\pair#1{\const{#1}}
 
\def\Stmt {\mathbf{Stmt}}
\def\Lab {\mathbf{Lab}}
\def\Blocks{\mathbf{Blocks}}
\def\Var {\mathbf{Var}}
 
 
\def\skip {\texttt{skip}\ }
\def\while{\texttt{while}\ }
\def\do {\texttt{do}\ }
\def\ifl {\texttt{if}\ }
\def\thenl {\texttt{then}\ }
\def\elsel {\texttt{else}\ }
\def\print{\texttt{print}\ }
\def\cont {\texttt{continue}\ }
\def\break{\texttt{break}\ }
 
\def\haskell{\textsc{Haskell}}
 
\def\program#1{\fbox{\begin{minipage}{0.5\textwidth}\protect{$\begin{array}{ll} #1 \end{array}$}\end{minipage}}}

\def\restabR#1#2[#3]{\begin{table}\label{#3}\begin{center}\caption{#1}\begin{sideways}\input{#2}\end{sideways}\end{center}\end{table}}

\def\restab#1#2[#3]{\begin{table}\label{#3}\begin{center}\caption{#1}\input{#2}\end{center}\end{table}}



\def\Tiny{\fontsize{3pt}{3pt}\selectfont}
\def\hs#1{\texttt{#1}}


 
\begin{document}
\author{Chris Eidhof, Rui S. Barbosa}
\title{Data Flow Assignment I \\ Automatic Program Analysis}
 
\maketitle
 
\section{Part 1}
 
Our analysis for Strongly Live Variables is almost the same as the Live
Variables analysis. We now only present the changes made. We
change our $gen$ function to take an extra argument $l$ of type $L = \pset{\Var_\star}$ which
corresponds to the set of strongly live variables at the exit of the corresponding block.
This is needed because the strong liveliness of a variable before
being used in an assignment block
depends on whether the assigned variable is strongly live at the exit of that block.
 
\begin{align*}
kill_{SLV} & \; : \; \Blocks_\star \to \pset{\Var_\star} \\
kill_{SLV}([x:=a]^\ell) & \eq \{x\} \\
kill_{SLV}([\skip]^\ell) & \eq \emptyset\\
kill_{SLV}([b]^\ell) & \eq \emptyset \\
\\
gen_{SLV} & \; : \; \Blocks_\star \times \pset{\Var_\star} \to \pset{\Var_\star} \\
gen_{SLV}([x:=a]^\ell,l) & \eq
         \begin{cases}
          FV(a) & \text{if $x \in l$} \\
          \emptyset & \text{otherwise}
         \end{cases} \\
gen_{SLV}([\skip]^\ell,l) & \eq \emptyset\\
gen_{SLV}([b]^\ell,l) & \eq FV(b)
\end{align*}
 
This means our $f_\ell$ also has to change:
 
\[ f_\ell(l) \eq (l \setminus kill([B]^\ell)) \cup gen([B]^\ell, l) \;\; \text{where} \;\; [B]^\ell
\in blocks(S_\star)
\]
 
Moreover, in our analysis, $\iota$ will not be necessarily empty. Instead, it will be a chosen set of
variables of interest.
By now, there are no means for a program
to communicate its results besides the inspection of some variables when execution halts: those variables
are what we call the variables of interest at the end of the program. Clearly,
those variables need to be considered live at the exit of all final blocks, so that we can always inspect their values.
When $\iota$ is empty, the only way to have intermediate
non-empty sets of strongly live variables will be by using them in conditional expressions (which control the
flow of the program). In that case, we are not interested in any output from the program.
 
 
\section{Part 2}
 
We developed some modules in \haskell which allow us to define and perform
data flow analysis on While programs. To define an analysis,
one needs to specify how to create a monotone framework from a program.
For the flow graph generation, one can use the functions \hs{forward} and \hs{backward}.
As for the transfer functions, also a few useful combinators were introduced, mainly
for the cases when the underlying lattice is a set: \hs{may} and \hs{must}
can be used to specify which operation to use and there is also special support
for the typical $gen/kill$ transfer functions (\hs{genkill}) and for
$gen/kill$ functions that can receive an additional parameter as in the case of SLV
(\hs{depgenkill}). Having defined the conversion from programs to monotone frameworks,
our modules automaticaly generate the equations, solve them through chaotic iteration
and show the result in \TeX table format. 

As an example, the definition of SLV will read as:

{\small
\begin{verbatim}
stronglivevariables i  = createDataFlowAnalyser backward (may (const i, const $ depgenkill (genSLV , const . killSLV)))
\end{verbatim}
}

\noindent
where the \hs{genSLV} and \hs{killSLV} are just as defined in the previous part. Then, we just define

{\small
\begin{verbatim}
doSLV iota = (resultToTable (\"SLV\",True,\"exit\",\"entry\") .
             scan_analyze (stronglivevariables S.empty) . labelProgram)
\end{verbatim}
}

\noindent
which is a function that given a set of variables of interest
and an unlabeled program, performs the strong live variable analysis and displays
the intermediate results in a $TeX$ table.


To demonstrate the Strong Live Variable Analysis, we will use this \haskell program
to perform chaotic iteration on the following simple example program

\label{exprog1}
\program{\input{prog.tex}}
 
Our iteration stabilizes at the 9th iteration if we take $y$ as our $\iota$:
 
\restabR{Strongly Live Variable Analysis with $\iota = \{r\}$}{resultr.tex}[resultr]

\restab{Strongly Live Variable Analysis with $\iota = \{y\}$}{resulty.tex}[resulty]

\restab{Strongly Live Variable Analysis with $\iota = \{a\}$}{resulta.tex}[resulta]

\restab{Strongly Live Variable Analysis with $\iota = \emptyset$}{resultempty.tex}[resultempty]
 
\section{Part 3}
 
We start by adding the new labelled constructs to the abstract syntax of the language:
\begin{align*}
 Stmt ::= \; & \ldots \\
 ~ |\;\; & [\print a]^\ell \\
 ~ |\;\; & [x_1,\ldots,x_n := a_1,\ldots,a_n]^\ell , n \in \N \\
 ~ |\;\; & [\cont]^\ell \\
 ~ |\;\; & [\break]^\ell
\end{align*}
 
We also extend the $blocks$ function accordingly in a pretty straightforward manner. The definition
of the $labels$ function given in the book stays valid.
\begin{align*}
 blocks([\print a]^\ell) & \eq \{[\print a]^\ell\} \\
 blocks([x_1,\ldots,x_n := a_1,\ldots,a_n]^\ell) & \eq \{[x_1,\ldots,x_n := a_1,\ldots,a_n]^\ell\} \\
 blocks([\cont]^\ell) & \eq \{[\cont]^\ell\} \\
 blocks([\break]^\ell) & \eq \{[\break]^\ell\}
\end{align*}
 
 
Then, for each of those constructs, we give a description of its semantics (the actual rules may be found
in table \ref{semantics}) and explain the modification that need to be done
to the monotone framework, particularly the functions defining the flow control
($init$,$final$ and $flow$) as well as the $gen_{SLV}$ and $kill_{SLV}$ functions
which define our analysis. We also give some example
programs in order to demonstrate the resulting analysis.
 
\subsection{Print}
 
Informally, the semantics of this construct will be to write the value of arithmetic expression $a$ to an output
stream $Z^\star$. When defining the formal semantics, we will add an extra component to the state corresponding to the
list of values printed so far.
 
This new construct has no effect on the flow control of the program. Hence, the related functions are
defined just as for the other simple statements (assignments and $\skip$).
\begin{align*}
init([\print a]^\ell) & = \ell \\
final([\print a]^\ell) & = \{\ell\} \\
flow([\print a]^\ell) & = \emptyset
\end{align*}
 
From the Strongly Live Variables Analysis point of view, the most import thing is
the added constraint that all the variables used in expression $a$ need to be
considered (strongly) live when entering the $[\print a]^\ell$ block. Hence, the $gen$ and $kill$ are extended in this way:
\begin{align*}
kill_{SLV}([\print a]^\ell) & \eq \emptyset \\
\\
gen_{SLV}([\print a]^\ell,l) & \eq FV(a)
\end{align*}
 
Note that the $\print$ construct provides the program with a new capability of conveying
results to the outside. Thus, we are not required to consider $\iota \neq \emptyset$ from now on.
We simply need to print the so called variables of interest at the end of our program.
For example, replacing the $\skip$ by  $[\print r]$ at the end of program \ref{exprog1},
we get the results in Table \ref{resultr_prt}. They are the same as Table \ref{resultr}
 
\program{\input{progprtr.tex}}

\restabR{Strongly Live Variable Analysis with $\iota = \{\}$ and $\print$}{resultr_prt.tex}[resultr_prt]
 
\subsection{Simultaneous Assignements}
 
The meaning of multiple assignments ($[v_1,\ldots,v_n := a_1,\ldots,a_n]^\ell$) is
pretty straightforward: all expressions in the right hand side are evaluated and
then each of them is assigned to the corresponding variable on the left hand side, in left to right order.
 
As in the previous case, the flow related functions are easily extended.
\begin{align*}
init ([v_1,\ldots,v_n = a_1,\ldots,a_n]^\ell) & \eq \ell \\
final([v_1,\ldots,v_n = a_1,\ldots,a_n]^\ell) & \eq \{\ell\} \\
flow ([v_1,\ldots,v_n = a_1,\ldots,a_n]^\ell) & \eq \emptyset
\end{align*}
 
Clearly, as with simple assignements, the $kill$
function just enumerates the variables which are assigned to in $\ell$.
 
On the other hand, the
$gen$ function needs to give all the variables required to be strongly live before the block.
As a first guess, we could include all the
free variables used in expressions assigned to variables which are strongly live after the block, closely following the
single assignement case. However, this solution is not optimal in the case where the $v_i$ are not pairwise distinct.
For example, considering the block $[x,y,x := a,b,c]^\ell$) with $x$ and $y$ strongly live after it, the variables in expression
$a$ need not be considered strongly live before the block, as the attribution $x:=a$ will be immediately overwritten
by $x:=c$ (recall that attributions are done in left to right order).
 
Therefore, we extend the $kill$ and $gen$ functions in the following manner:
\begin{align*}
kill_{SLV}([v_1,\ldots,v_n := a_1,\ldots,a_n]^\ell) & \eq \bigcup\{\{v_i\} | 1 \leq i\leq n\} \\
\\
gen_{SLV}([v_1,\ldots,v_n := a_1,\ldots,a_n]^\ell,l) & \eq \bigcup\{FV(a_i) | v_i \in l \wedge \neg \exists j : j>i \,.\, v_j = v_i\}
\end{align*}
 
To demonstrate the use of this construct, let us consider a slightly different version of
\ref{exprog1}, where some assignments have been grouped togheter. Also, some new (dummy)
assignments were added to illustrate the situation when multiple assignments to the same
variable occur.

\program{\input{progmassr.tex}}

If we now do Strongly Live Variable Analysis on this program, we get the results in Table \ref{resultr_mass}.

\restabR{Strongly Live Variable Analysis with $\iota = r$ and multiple assignments}{resultr_mass.tex}[resultr_mass]
 
\subsection{Break and Continue}
The meaning of these constructs inside while loops is just what we are used to:
a $\break$ statement cause the program to jump immediately out of the loop whilst
a $\cont$ statement \emph{restarts} the while loop execution (including testing the condition).
If the constructs are found outside of a while loop, the compiler/interpreter should give an error before doing the analysis (it is a simple syntactic check).
 
The formal semantics could be given in a way resembling exception handling in Java (where while loops take
the r\^{o}le of catch), as we briefly explain. These constructs
would cause execution to halt in a new kind of state
enclosing the information of the \emph{then current} state along with an annotation of the reason
for halting ($\break$ or $\cont$).
We would then need several rules to deal with those halting states.
Firstly, we would need to propagate the halting states unchanged through consecutive statements:
sequencing ($;$) with another statement would ignore that second statement and enclosing
if-then-else constructs would also propagate the halting state (for this second case, the current
rule suffices). Finally, the way to deal with $\while$ should also change. A possibility
would be to change the $[wh_1]$ rule so that it takes a look similar to that of $[seq_1]$.
Then, if the body statement results in a $\cont$-annotated state, the annotation would be dropped
and the while loop restarted. If the state was $\break$-annotated, the annotation would also be
dropped but the while loop would be considered finished.
 
It is worth noticing that if a computation ends in an annotated state and there is
no enclosing while loop, the program itself as a whole would end in that kind of state.
We will consider that to be a error terminating program. Our framework will not consider
the top-level $\break$ and $\cont$ statements to be part of the final statements of
a program. That means that we do not care about the values of the variables of interest
if the program terminates in that way. 
On the other hand, this assumption will facilitate the definition of
the flow related functions: for example, when considering that a $\cont$ statement is not
a final statement of $S_1$, then the definition of $flow(S_1;S_2)$ will not consider the flow
from the $\cont$ statement to $init(S_2)$, just like one would expect. We then need to account
for the existence of these constructs in the definition of the flow information functions
for $\while$ statements (as well as the $\break$ and $\cont$, of course).
Those definitions now read:
\begin{align*}
init ([\cont]^\ell) &\eq \ell & init ([\break]^\ell) &\eq \ell \\
final([\cont]^\ell) &\eq \emptyset & final([\break]^\ell) &\eq \emptyset \\
flow ([\cont]^\ell) &\eq \emptyset & flow ([\break]^\ell) &\eq \emptyset
\end{align*}
\begin{eqnarray*}
init (\while [b]^\ell \do S) &\eq& \ell \\
final(\while [b]^\ell \do S) &\eq& \ell \cup breaksOf(S) \\
flow (\while [b]^\ell \do S) &\eq& \{(l,init(S))\} \cup flow(S) \\
                            & & \cup\; \{(l',l) | l' \in final(S)\} \\
                            & & \cup\; \{(l',l) | l' \in continuesOf(S)\}
\end{eqnarray*}
where $continuesOf$ and $breaksOf$ functions are auxiliary functions which map a given program statement
to the labels of its top level (not nested in a $\while$) $\cont$ and $\break$ statements, respectively:
 
\begin{align*}
continuesOf & \; : \; \Stmt \to \pset{\Lab} \\
continuesOf([x:=a]^\ell) & \eq \emptyset \\
continuesOf([\skip]^\ell) & \eq \emptyset \\
continuesOf([x_1,\ldots,x_:=a_1,\ldots,a_n]^\ell) & \eq \emptyset \\
continuesOf([\print a]^\ell) & \eq \emptyset \\
continuesOf([\cont]^\ell) & \eq \{\ell\} \\
continuesOf([\break]^\ell) & \eq \emptyset \\
continuesOf(S_1;S_2) & \eq continuesOf(S_1) \cup continuesOf(S_2) \\
continuesOf(\ifl [b]^\ell \thenl S_1 \elsel S_2) & \eq continuesOf(S_1) \cup continuesOf(S_2) \\
continuesOf(\while [b]^\ell \do S) & \eq \emptyset \\
\\
breaksOf & \; : \; \Stmt \to \pset{\Lab} \\
breaksOf([x:=a]^\ell) & \eq \emptyset \\
breaksOf([\skip]^\ell) & \eq \emptyset \\
breaksOf([x_1,\ldots,x_:=a_1,\ldots,a_n]^\ell) & \eq \emptyset \\
breaksOf([\print a]^\ell) & \eq \emptyset \\
breaksOf([\cont]^\ell) & \eq \emptyset \\
breaksOf([\break]^\ell) & \eq \{\ell\} \\
breaksOf(S_1;S_2) & \eq breaksOf(S_1) \cup breaksOf(S_2) \\
breaksOf(\ifl [b]^\ell \thenl S_1 \elsel S_2) & \eq breaksOf(S_1) \cup breaksOf(S_2) \\
breaksOf(\while [b]^\ell \do S) & \eq \emptyset
\end{align*}
 
We presented above the changes made on the definition of the flow of the program.
These definitions can be used in all monotone frameworks. Regarding
the Strongly Live Variable analysis, these constructs are easy to handle. They
are just like $\skip$ as they do not change the state of the data whatsoever (in the
semantics informally presented above, the state was frozen with an annotation and then de-annotated
when execution restarts at the right program point. Thus, we have:
\begin{align*}
kill_{SLV}([\cont]^\ell) \eq kill_{SLV}([\break]^\ell) & \eq \emptyset \\
\\
gen_{SLV} ([\cont]^\ell,l) \eq gen_{SLV} ([\break]^\ell,l) & \eq \emptyset
\end{align*}
 
We can now analyze two example programs. First, we will do Strongly Live Variable Analysis on the following program:
 
\program{\input{progbc.tex}}

The results are in Table \ref{result_bc}.

\begin{table}
\label{result_bc}
\caption{Strongly Live Variable Analysis with $\break$ and $\cont$}
\begin{sideways}
\input{result_bc.tex}
\end{sideways}
\end{table}


\program{\input{progbc2.tex}}



In our operation semantics, we now also have an out variable that captures what has been printed so far. It is just a list of $Z$, and when doing a $\print a$ we append $a$ to that list.
 
\begin{table}\label{semantics}
\caption{Modifications to the Operational Semantics}
{\small
\begin{eqnarray*}
\text{state} & \sigma = (\eta, out) \in (Var \to \Z) \times \Z^\star & \\
& & \\
& & \\
\pair{print}\;\; &
 \pair{\print a, \sigma = (\eta,out)} \longrightarrow (\eta, snoc(out,\A{a}\eta))\; &\\
& & \\
\pair{multass}\;\; &
 \pair{x_1,\ldots,x_n := a_1,\ldots,a_n, \sigma = (\eta,out)} \longrightarrow
 (\eta[x_1\mapsto \A{a_1}\eta]\ldots[x_n\mapsto \A{a_n}\eta] , out)\; &\\
& & \\
\pair{cont\_fire}\;\; &
 \pair{\cont, \sigma} \longrightarrow [\sigma]^{continue} \; &\\
& & \\
\pair{break\_fire}\;\; &
 \pair{\break, \sigma} \longrightarrow [\sigma]^{break} \; &\\
& & \\
\pair{cont\_prop}\;\; &
\begin{prooftree}
\pair{S_1,\sigma} \longrightarrow [\sigma']^{continue}
\justifies
\pair{S_1 ; S_2,\sigma} \longrightarrow [\sigma']^{continue}
\thickness=0.08em
\end{prooftree}\\
& & \\
\pair{break\_prop}\;\; &
\begin{prooftree}
\pair{S_1,\sigma} \longrightarrow [\sigma']^{break}
\justifies
\pair{S_1 ; S_2,\sigma} \longrightarrow [\sigma']^{break}
%\thickness=0.08em
\end{prooftree}\\
& & \\
\pair{cont\_catch}\;\; &
\begin{prooftree}
\pair{S,\sigma} \longrightarrow [\sigma']^{continue}
\justifies
\pair{\while [b]^\ell \do S,\sigma} \longrightarrow \pair{\while [b]^\ell \do S,\sigma'}
\thickness=0.08em
\end{prooftree}\\
& & \\
\pair{break\_catch}\;\; &
\begin{prooftree}
\pair{S,\sigma} \longrightarrow [\sigma']^{break}
\justifies
\pair{\while [b]^\ell \do S,\sigma} \longrightarrow \sigma
\thickness=0.08em
\end{prooftree}\\
\end{eqnarray*}
}
\end{table}
 
\end{document}
 
  
