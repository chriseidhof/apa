\documentclass[a4wide,12pt]{article}
\usepackage{a4wide}
\usepackage{times}
\usepackage{fancyvrb}
\usepackage{url}
\usepackage{enumerate}
\usepackage{palatino}
\usepackage{rotating}
 
%\usepackage{prooftree}

\usepackage{amsmath, amsthm, amssymb}

\theoremstyle{definition}
\newtheorem{defi}{Definition}
\newtheorem{example}{Example}
\newtheorem*{conj}{Conjecture}
\newtheorem*{prob}{Problem}
\newtheorem*{question}{Question}
\theoremstyle{plain}
\newtheorem{theo}{Theorem}
\newtheorem{prop}[theo]{Proposition}
\newtheorem{lemma}[theo]{Lemma}
\newtheorem{cor}[theo]{Corollary}
\newtheorem*{theo*}{Theorem}
\newtheorem*{prop*}{Proposition}
\newtheorem*{lemma*}{Lemma}
\newtheorem*{cor*}{Corollary}
\theoremstyle{remark}
\newtheorem*{remark}{Remark}
\newtheorem*{notation}{Notation}
\def\qed{\begin{flushright} $\Box$ \end{flushright}}

\newenvironment{prf}
                {\vspace{-2mm} \noindent {\bf Proof.}}
                {\par \nopagebreak \qed }
\newenvironment{namedprf}[1]
                {\noindent {\bf Proof (#1).}}
                {\par \nopagebreak \qed }


\def\logequiv{\Leftrightarrow}

\allowdisplaybreaks[0]
 
\def\eq{\;\; = \;\;}
\def\N{\mathbb{N}}
\def\Z{\mathbb{Z}}
 
\def\pset#1{\mathcal{P}(#1)}
\def\A#1{\mathcal{A}[\hspace{-1pt}[#1]\hspace{-1pt}]}
 
\def\const#1{\mathopen{\langle}#1\mathclose{\rangle}} % <a,b,...z>
\def\pair#1{\const{#1}}
 
\def\Stmt {\mathbf{Stmt}}
\def\Lab {\mathbf{Lab}}
\def\Blocks{\mathbf{Blocks}}
\def\Var {\mathbf{Var}}
 
 
\def\skip {\texttt{skip}\ }
\def\whilel{\texttt{while}\ }
\def\dol {\texttt{do}\ }
\def\ifl {\texttt{if}\ }
\def\thenl {\texttt{then}\ }
\def\elsel {\texttt{else}\ }
\def\printl{\texttt{print}\ }
\def\contl {\texttt{continue}\ }
\def\breakl{\texttt{break}\ }
 
\def\haskell{\textsc{Haskell}}
\def\starto{\overset{\star}{\to}}
 
\newcounter{Progenvcount}
\setcounter{Progenvcount}{0}
\newenvironment{progenv}
                {\refstepcounter{Progenvcount} \nopagebreak
                 \bigskip\hrule\nopagebreak\medskip\noindent
                 {\bf Program \arabic{Progenvcount}.} \nopagebreak\vspace{0.3cm} \nopagebreak \\ \nopagebreak
                  \nopagebreak
                 $\begin{array}{ll}}
                {\end{array}$ \bigskip\hrule\bigskip\bigskip }


%\def\programold#1{\fbox{\begin{minipage}{0.5\textwidth}\protect{$\begin{array}{ll} #1 \end{array}$}\end{minipage}}}

\def\restabR#1#2[#3]{
\begin{table}
{\scriptsize
\caption{#1}\label{#3}
\begin{center}\begin{sideways}\input{#2}\end{sideways}\end{center}
}
\end{table}}

\def\restabRtiny#1#2[#3]{
\begin{table}
{\tiny
\caption{#1}\label{#3}
\begin{center}\begin{sideways}\input{#2}\end{sideways}\end{center}
}
\end{table}}


\def\restab#1#2[#3]{
\begin{table}
\caption{#1}\label{#3}
\begin{center}\input{#2}\end{center}
\end{table}}


\def\program#1[#2]{\begin{progenv}\label{#2}\input{#1}\end{progenv}}

\def\Tiny{\fontsize{3pt}{3pt}\selectfont}
\def\hs#1{\texttt{#1}}

\def\rdsub{RD^\subseteq(S_\star)}

 
\begin{document}

\section{~}
We define the set $RD^\subset(S)$ of constraints generated for a program $S$,
not taking in account introducing undefinedness of all variables in the beginning. 
\begin{align} 
\label{def_constr}
RD^{\subseteq}(S) = & \; \{X(\ell) \supseteq N(\ell) \setminus kill_{RD}(B^l)) \cup gen_{RD}(B^l) \mid B^l \in blocks(S)\} \\
\cup & \; \{N(\ell) \supseteq X(\ell') \mid (\ell', \ell) \in flow(S) \}
\end{align}


Now, the subprogram property (lemma 2.16 in the book ) holds and we can easily prove the lemmas
and the main theorem, then getting to the following result:

\begin{cor}\label{corr}
If $reach \models RD^\subseteq(S)$ (with $S$ being label consistent) then:

\begin{enumerate}
\item If $\pair{S, \sigma, tr} \starto \pair{S', \sigma', tr'}$ and $ tr \sim
N(init(S))$ then $tr' \sim N(init(S'))$
\item If $\pair{S, \sigma, tr} \starto \pair{\sigma', tr'}$ and $ tr \sim
N(init(S))$ then $tr' \sim X(\ell)$ for some $\ell \in final(S)$.
\end{enumerate}
\end{cor}

\section{~}

Informally, the above result means that execution of a piece of code keeps the analysis results correct with respect to the semantics.
That is, if we have $reach \models RD^\subseteq(S)$ and we have a correct analysis result at $live_entry(init(S))$, then the analysis results will
be correct with respect to every possible path at all (reachable) program points. 
This is exactly what one would expect from a solution to $RD^\subseteq(S)$. 

We now consider the whole $RD$ analysis:
for a program $S_\star$, we generate all the constraints we had plus a constraint
introducing variable undefinedness at the beginning. 
 
\[RD^\subseteq_\star(S_\star) = RD^\subseteq(S_\star) \cup \{ \{(x,?) \mid x \in \Var_\star\} \subseteq init(S_\star) \}\]

This set of constraints is intended to be applied only once, for the ``full'' program.
The idea is that it correctly captures the semantics of executions of complete programs
(but not of subprograms where some code has already been executed, unless we wanted to ``forget''
this code).

In the semantics side, a full evaluation of $S_\star$ starts from $\pair{S_\star,\sigma,tr_0}$
where $\sigma$ is some initial state and $tr_0$ is the trace at the beginning, where yet no code
has been executed.
\[tr_0 = [ (x,?) \mid x \in Var_\star(S_\star) ] \]

With these definitions, we can then formulate our correctness result.
\begin{theo}
If $reach \models RD_\star^\subseteq(S)$ (with $S$ being label consistent) then:

\begin{enumerate}
\item If $\pair{S, \sigma_0, tr_0} \starto \pair{S', \sigma', tr'}$ then $ tr' \sim
N(init(S'))$ 
\item If $\pair{S, \sigma_0, tr_0} \starto \pair{\sigma', tr'}$ then $ tr' \sim
X(\ell)$ for some $\ell \in final(S)$.
\end{enumerate}
\end{theo}
 
The proof is straightforward: basically the old constraints guarantee that the results are correctly propagated to each (reachable)
program point (as seen in \ref{corr}) while the new constraint give us correct results to start with.

\end{document}
