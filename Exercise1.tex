\documentclass[a4wide,12pt]{article} 
\usepackage{a4wide}
\usepackage{times}
\usepackage{fancyvrb}
\usepackage{url}
\usepackage{enumerate}
\usepackage{palatino}
%include polycode.fmt 
%options ghci

\def\skip{\texttt{skip}}
\def\eq{\;\; =  \;\;}
\usepackage{amsmath, amsthm, amssymb}

\begin{document}
\author{Chris Eidhof, Rui S. Barbosa}
\title{Assignment 1}

\maketitle

\section{Part 1}

% TODO ask Jurriaan if he wants the bottom part of table 2.4 as well, or just
% the $f_\ell$

Our analysis for Strongly Live Variables is almost the same as the Live
Variables analysise. We now only present the functions that have changed. We
change our $gen$ function to take an extra argument $l$ of type $L$ which
corresponds to the set strongly live variables at the exit of the block.

$kill$ and $gen$ functions
\begin{align*}
kill_{SLV}([x:=a]^\ell) &  \eq \{x\} \\
kill_{SLV}([\skip]^\ell) & \eq \emptyset\\
kill_{SLV}([b]^\ell) &  \eq \emptyset \\
\\
gen_{SLV}([x:=a]^\ell,l) & \eq
         \begin{cases}
          FV(a) & \text{if $x \in l$} \\
          \emptyset & \text{otherwise}
         \end{cases} \\
gen_{SLV}([\skip]^\ell,l) & \eq \emptyset\\
gen_{SLV}([b]^\ell,l) & \eq FV(b)
\end{align*}

This means our $f_\ell$ also has to change:

\[ f_\ell(l) \eq (l \setminus kill([B]^\ell)) \cup gen([B]^\ell, l) \;\; \text{where} \;\; [B]^\ell
\in blocks(S_\star)
\]

In our analysis, $\iota$ will not be necessarily empty, it will represent the
variables of interest. If it is empty, the only way to have intermediate
non-empty variables of interest will be by using them in conditions.

\end{document}


