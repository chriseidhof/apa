\documentclass[a4wide,12pt]{article}
\usepackage{a4wide}
\usepackage{times}
\usepackage{fancyvrb}
\usepackage{url}
\usepackage{enumerate}
\usepackage{palatino}
\usepackage{rotating}
 
%\usepackage{prooftree}

\usepackage{amsmath, amsthm, amssymb}

\theoremstyle{definition}
\newtheorem{defi}{Definition}
\newtheorem{example}{Example}
\newtheorem*{conj}{Conjecture}
\newtheorem*{prob}{Problem}
\newtheorem*{question}{Question}
\theoremstyle{plain} 
\newtheorem{theo}{Theorem}
\newtheorem{prop}[theo]{Proposition}
\newtheorem{lemma}[theo]{Lemma}
\newtheorem{cor}[theo]{Corollary}
\newtheorem*{theo*}{Theorem}
\newtheorem*{prop*}{Proposition}
\newtheorem*{lemma*}{Lemma}
\newtheorem*{cor*}{Corollary}
\theoremstyle{remark}
\newtheorem*{remark}{Remark}
\newtheorem*{notation}{Notation}
\def\qed{\begin{flushright} $\Box$ \end{flushright}}

\newenvironment{prf}
                {\vspace{-2mm} \noindent {\bf Proof.}}
                {\par \nopagebreak \qed }
\newenvironment{namedprf}[1]
                {\noindent {\bf Proof (#1).}}
                {\par \nopagebreak \qed }


\def\logequiv{\Leftrightarrow}

\allowdisplaybreaks[0]
 
\def\eq{\;\; = \;\;}
\def\N{\mathbb{N}}
\def\Z{\mathbb{Z}}
\def\compose{\circ}
\def\Zext{\Z^\top_\bot}
 
\def\pset#1{\mathcal{P}(#1)}
\def\A#1{\mathcal{A}[\hspace{-1pt}[#1]\hspace{-1pt}]}
 
\def\const#1{\mathopen{\langle}#1\mathclose{\rangle}} % <a,b,...z>
\def\pair#1{\const{#1}}
 
\def\Stmt {\mathbf{Stmt}}
\def\Lab {\mathbf{Lab}}
\def\Labstar {\mathbf{Lab_\star}}
\def\Blocks{\mathbf{Blocks}}
\def\Var {\mathbf{Var}}
\def\Varstar {\mathbf{Var_\star}}
 
 
\def\skip {\texttt{skip}\ }
\def\whilel{\texttt{while}\ }
\def\dol {\texttt{do}\ }
\def\ifl {\texttt{if}\ }
\def\thenl {\texttt{then}\ }
\def\elsel {\texttt{else}\ }
\def\printl{\texttt{print}\ }
\def\contl {\texttt{continue}\ }
\def\breakl{\texttt{break}\ }
 
\def\haskell{\textsc{Haskell}}
\def\starto{\overset{\star}{\to}}

\newcounter{Progenvcount}
\setcounter{Progenvcount}{0}
\newenvironment{progenv}
                {\refstepcounter{Progenvcount} \nopagebreak
                 \bigskip\hrule\nopagebreak\medskip\noindent
                 {\bf Program \arabic{Progenvcount}.} \nopagebreak\vspace{0.3cm} \nopagebreak \\ \nopagebreak
                  \nopagebreak
                 $\begin{array}{ll}}
                {\end{array}$ \bigskip\hrule\bigskip\bigskip }


%\def\programold#1{\fbox{\begin{minipage}{0.5\textwidth}\protect{$\begin{array}{ll} #1 \end{array}$}\end{minipage}}}

\def\restabR#1#2[#3]{
\begin{table}
{\scriptsize
\caption{#1}\label{#3}
\begin{center}\begin{sideways}\input{#2}\end{sideways}\end{center}
}
\end{table}}

\def\restabRtiny#1#2[#3]{
\begin{table}
{\tiny
\caption{#1}\label{#3}
\begin{center}\begin{sideways}\input{#2}\end{sideways}\end{center}
}
\end{table}}


\def\restab#1#2[#3]{
\begin{table}
\caption{#1}\label{#3}
\begin{center}\input{#2}\end{center}
\end{table}}


\def\program#1[#2]{\begin{progenv}\label{#2}\input{#1}\end{progenv}}

\def\Tiny{\fontsize{3pt}{3pt}\selectfont}
\def\hs#1{\texttt{#1}}
 
\begin{document}
\author{Chris Eidhof, Rui S. Barbosa}
\title{Abstract Interpretation Assignment \\ Automatic Program Analysis}
 
\maketitle

\section~{Galois Connections}

We start out by giving a pair of functions $\alpha, \gamma$ that transform
$\pset{\Z}$ into $\Zext$ and the other way around:

\begin{eqnarray*}
\alpha & \emptyset & = \bot \\
\alpha & \{n\} & = n \\
\alpha & otherwise & = \top \\
& & \\
\gamma & \bot & = \emptyset \\
\gamma & n & = \{n\} \\
\gamma & \top & = \Z 
\end{eqnarray*}

TODO chris: step by step.

If we take $\alpha \compose \gamma$ we can verify for all values in $\Zext$ that
this is equal to the identity function. For $\gamma \compose \alpha$ we have
that in the case of $\emptyset$ and a singleton set that it is the identity
function, but when applying $\gamma \compose \alpha$ to a non-empty set with more
than one element we get $\Z$ as a result. Thus, this is clearly not the identity
function, but the relation $gamma \compose \alpha \sqsubseteq id$ holds, which
makes $start = (\pset{\Z}, \alpha, \gamma, \Zext)$ a Galois insertion.

We can now apply a number of transformations to end up in our final result. These
transformations preserve the Galois insertion and thus the Galois connection.
Firstly, we will transform $start$ using the total function space combinator,
yielding the Galois insertion $(\Varstar \to \pset{\Z}, \alpha^1, \gamma^1,
\Varstar \to \Zext)$. Next, we use the Galois Connection from slide 13 in
lecture 13, which gives us $(\pset{\Varstar \to \Z}, \alpha^2, \gamma^2,
\Varstar \to \pset{\Z})$. If we then apply composition we get 
$(\pset{\Varstar \to \Z}, \alpha^3, \gamma^3, \Varstar \to \Zext)$. Now we only
have apply the total function space combinator once more to get to our final
result, $(\pset{\Labstar \to \Varstar \to \Z}, \alpha^4, \gamma^4, \Labstar \to \Varstar \to \Zext)$

ii: Together
iii: Rui
iv: Chris, correction by Rui.

\[diff(x,y) = lenght(x)-lenght(y)\]
\[ range \comp diff \]

\[cmp(x,y) = \{S|isSuffix(x,y)\} \cup \{H|sameHead(x,y) \]

v: Chris
vi: Rui

\end{document}  

